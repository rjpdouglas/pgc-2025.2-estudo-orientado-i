\documentclass[12pt]{article}

\usepackage{sbc-template}
\usepackage{graphicx,url}
\usepackage[utf8]{inputenc} 
\usepackage[portuguese]{babel}
\usepackage{placeins}
\usepackage{makecell}
\usepackage{xcolor}

\sloppy

\title{Proposta de estudo orientado: Um modelo explicável para classificação de arritmias cardíacas a partir de sinais de ECG com uso de arquitetura \textit{Transformer}}

\author{Orientadora: Taiane Coelho Ramos\inst{1}, Orientando: Douglas Blanc Pereira\inst{1}}

\address{Instituto de Informática -- Universidade Federal Fluminense
  (UFF)\\
  Av. Gal. Milton Tavares de Souza, s/n - São Domingos, Niterói - RJ, 24210-310
  \email{dpereira@id.uff.br, taiane\_ramos@id.uff.br}   
}
\begin{document} 

\maketitle

\section{Descrição do tema}

As arritmias cardíacas representam anomalias nos impulsos elétricos do coração, resultando em alterações do ritmo cardíaco que podem ser indicativas de doenças potencialmente fatais. As doenças cardiovasculares são a principal causa de morte no Brasil e uma das líderes em óbitos globais, tornando o diagnóstico precoce um fator crucial para a redução da mortalidade. O eletrocardiograma (ECG) é o exame primário para essa investigação, devido ao seu baixo custo e rapidez \cite{zirpolo2025modelo,da2024arritmias}. Porém, a análise de sinais de ECG enfrenta desafios significativos, como a necessidade de interpretação por especialistas, a suscetibilidade a ruídos e a baixa sensibilidade para certas condições, o que pode levar a divergências diagnósticas. Para mitigar essas limitações, sistemas automatizados baseados em aprendizado profundo (Deep Learning) têm demonstrado grande potencial, oferecendo análises mais precisas e assertivas \cite{ansari2023deep}.

A análise de dados sequenciais, como os sinais de ECG para, tem sido dominada por arquiteturas de Redes Neurais Recorrentes (RNNs), como a \textit{Long Short-Term Memory} (LSTM). A capacidade dessas redes de processar informações de forma sequencial, mantendo um estado que captura dependências temporais, tornou-as o estado da arte em diversas tarefas de modelagem de sequências. Entretanto, estudos mais atuais demonstram desempenho promissor da arquitetura \textit{Transformer}, que dispensa recorrência e se baseia unicamente em mecanismos de atenção para estabelecer dependências entre entrada e saída \cite{vaswani2017attention}. Esta arquitetura tem se mostrado eficaz em diversas tarefas de classificação de séries temporais, principalmente devido à sua capacidade de capturar dependências de longo prazo \cite{wen2022transformers}.

O objetivo deste estudo é realizar uma revisão de literatura com a finalidade de adquirir conhecimento sobre o estado da arte no tema, buscando problemas em aberto que envolvam o uso de arquitetura \textit{Transformer} e suas distribuições de atenção para gerar uma representação visual. Esta representação destacaria os segmentos do sinal que mais influenciaram a predição do modelo. A abordagem se alinha ao desenvolvimento de sistemas de Apoio ao Diagnóstico Assistido por Computador (CAD), onde a visualização pode servir como um suporte para a análise do especialista (o médico), aumentando a confiança e a transparência do modelo ao fornecer uma justificativa visual para ele. Por fim, será produzida uma proposta de dissertação sobre o tema.

\section{Resultados esperados} 

\begin{itemize}
  
  \item Revisão de literatura sobre a aplicação de arquiteturas \textit{Transformer} para a classificação de séries temporais, com foco em sinais biomédicos como o ECG.

  \item Explorar o uso de atenção para explicabilidade.

  \item Um protótipo do modelo \textit{Transformer} e obtenção de resultados preliminares.

  \item Uma proposta de pesquisa de dissertação.

  \item Redação de um esboço de artigo a ser submetido no semestre seguinte.
  
\end{itemize} 

\section{Critérios de avaliação} 

A avaliação será baseada nos seguintes itens:
\begin{enumerate}   

\item Cumprimento das etapas do estudo dentro dos prazos definidos pela professora conforme registrado nas tarefas do \textit{classroom} criado para o acompanhamento do estudo; 
\item Qualidade e clareza do texto de projeto de dissertação incorporando as modificações e correções solicitadas pela professora; 
\item Avaliação dos resultados parciais obtidos pelo protótipo do modelo descrito no esboço de artigo apresentado ao final desta disciplina.
\end{enumerate}

\section{Cronograma de atividades}

O orientando se reunirá semanalmente com a orientadora por uma hora. O acompanhamento das atividades será realizado via \textit{classroom}.

\begin{table}[!ht]
\caption{Cronograma das atividades (Semanas 1 a 13)}

\begin{tabular}{|l|c|c|c|c|c|c|c|c|c|c|c|c|c|}
\hline

\textbf{Atividade} & \textbf{1} & \textbf{2} & \textbf{3} & \textbf{4} & \textbf{5} & \textbf{6} & \textbf{7} &  \textbf{8} &  \textbf{9} & \textbf{10} & \textbf{11} & \textbf{12} & \textbf{13}  \\ 
  \hline
  \makecell[l]{Elaborar proposta de estudo \\ orientado} & X & & & & & & & & & & & & \\
  \hline
  \makecell[l]{Selecionar artigos sobre o \\ tema para revisão de literatura} & & X & X & & & & & & & & & & \\
  \hline
  \makecell[l]{Identificar estado da arte na \\ detecção automática de \\ arritmias cardíacas} & & & & X & & & & & & & & & \\
  \hline     
  \makecell[l]{Identificar questões abertas \\ da literatura sobre o tema} & & & & & X & & & & & & & & \\
  \hline
  \makecell[l]{Desenvolver protótipo de \\ modelo de classificação de \\ arritmias usando \textit{Transforer}} & & & & & & X & X & X & & & & & \\
  \hline
  \makecell[l]{Analisar resultados obtidos} & & & & & & & & & X & & & & \\
  \hline
  \makecell[l]{Esboçar artigo sobre o desen-\\volvimento do protótipo e \\ resultados preliminares} & & & & & & & & & & X & X & & \\
  \hline
  \makecell[l]{Redigir proposta de pesquisa \\ de dissertação} & & & & & & & & & & & & X & X \\
  \hline
 
\end{tabular}
\label{tab:cronograma}
\end{table} 

\FloatBarrier
\bibliographystyle{sbc}
\bibliography{sbc-template} 

\vfill

\begin{flushright}
Niterói, \today
\end{flushright}

\vspace{4cm}

\begin{center}
    \begin{minipage}{0.6\textwidth}
        \centering
        \rule{\linewidth}{0.4pt} \\
        Dra. Taiane Coelho Ramos (orientadora)
    \end{minipage}
\end{center}

\vspace{4cm}

\begin{center}
    \begin{minipage}{0.6\textwidth}
        \centering
        \rule{\linewidth}{0.4pt} \\
        Douglas Blanc Pereira (orientando)
    \end{minipage}
\end{center}
\end{document}